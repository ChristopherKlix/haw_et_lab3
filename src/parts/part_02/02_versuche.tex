\part{Versuche}

\subfile{chapter_00/00_widerstandsmessung}

\chapter{Auswertung}

Die Zielsetzung konnte anhand der Experimente gut erfüllt werden.
Es sind Erkenntnisse über die sinnvolle Anwendung der Messmethoden gewonnen worden und das Verständnis von Ersatzquellen konnte durch Abgleichen der Messung mit den Berechnungen gefestigt werden.

Die Auswertung aus den Versuchen führen zu folgenden Erkenntnissen:
\begin{itemize}
  \item Eine Messung von kleinen Widerständen mittels Multimeter ist sehr ungenau.
  \item Leiterwiderstände haben einen großen Einfluss bei kleinen Widerständen.
  \item Analoge Multimeter sind deutlich fehleranfälliger als Digitale Multimeter.
  \item Stromrichtiges Messen ist besser geeignet für das Messen von großen Widerständen.
  \item Spannungsrichtiges Messen ist besser geeignet für das Messen von kleinen Widerständen.
  \item Zwei-Punkt-Methode sorgt bei kleinen Widerständen für sehr große Abweichung, die Vier-Punkt-Methode ist besser geeignet.
  \item Zwei- und Vier-Punkt-Methode verhalten sich beim Messen des 1kOHM Widerstandes nahezu identisch.
  \item Nichtlineare Widerstände stimmen nur bei sehr kleinen Strömen mit den Messungen eines Ohmmeters überein.
  \item Temperatur hat hier einen Einfluss auf den nichtlinearen Widerstand.
  \item Eine Schaltung kann durch die Berechnung des Innenwiderstands in eine Ersatzschaltung umgebaut werden.
  \item Die Messwerte in der Ersatzschaltung sind identisch zur Originalschaltung.
\end{itemize}
Für die nächsten Praktika empfiehlt es sich, parallel zur Versuchsdurchführung, ein Gruppenmitglied zur direkten Diagramm Erstellung zu beauftragen, um mögliche Messfehler oder anderweitige Abweichungen schnell erkennen zu können.
Zudem können sofort erste Erkenntnisse gewonnen werden, was die Erstellung des Berichtes vereinfacht.
