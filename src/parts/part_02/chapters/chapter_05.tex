\chapter[Wägeeinrichtung (Vollbrücke)]{Aufbau einer Wägeeinrichtung mit dem Biegestab (Vollbrücke)}

Im folgenden Versuch wird durch einen parallel geschalteten Widerstand \( R_a \)
bei einer Wheatstone-Brückenschaltung der Nullpunktfehler kompensiert
und so ein exakter Nullabgleich ermöglicht.
Dieser Fehler tritt meist bei Vollbrücken auf, die im Ausschlagverfahren betrieben werden,
bedingt durch die Toleranten der Widerstände.
Die Versorgungspannung \( U_0 \) wird wieder auf \( 6\si{\volt} \) eingestellt
und die Brückenspannung \( U_{ab} \) wird mit dem Digitalmultimeter \textit{METRAHit TECH} gemessen.

\begin{figure}[!h]\centering
    \vspace*{0.7cm}
    \begin{circuitikz}[american, scale = 0.7]
    \draw
    (0,0) to[european voltage source, l=$U_0$] (0,6)

    (0,6) to[short, -*] (6,6)
          to[variable resistor, l=$R_1$] (6,3)

    (6,6) to[short, -] (12,6)
          to[variable resistor, l=$R_3$] (12,3)

    (6,3) to[short, *-o] (7,3)
    (7,3) node[label=$a$] {}
    (12,3) to[short, *-o] (11,3)
    (11,3) node[label=$b$] {}

    (7,3) to[voltmeter, l=$U_{ab}$] (11,3)

    (6,3) to[variable resistor, l=$R_2$] (6,0)
    (12,3) to[variable resistor, l=$R_4$] (12,0)

    (12,3) to[short, -] (16,3)
           to[variable resistor, l=$R_a$] (16,0)
           to[short, -*] (12,0)

    (0,0) to[short, -*] (6,0)
          to[short, -*] (12,0)

    % (2.5,6.2) node[label={[font=\footnotesize]east:$I_M$}] {}
    % (4.3,4) node[label={[font=\footnotesize]east:$U_M$}] {}

    % (7.5,4) node[label={[font=\footnotesize]east:Long distance}] {}
    ;
    \end{circuitikz}
    \caption{Wägeeinrichtung (Vollbrücke)} \label{cir:full-bridge}
\end{figure}

\section[Messung]{Messung}

Mit Hilfe einer Präzisions-Widerstandsdekade ist der Widerstand \( R_a \)
so einzustellen das die Ausgleichsbedingung \( U_{ab} = 0 \si{\volt} \) erfüllt ist.
Um den richtigen Widerstand zu finden zu dem \( R_a \) parallel geschalten werden muss,
wird geschaut bei welchem Widerstand die Anzeige einen kleinen Ausschlag
oder einen Polaritätswechsel durch den Nulldurchgang anzeigt.
Im nächsten Schritt wird am freischwingenden Ende des Biegebalkens die Last erhöht
und parallel dazu die Brückenspannung \( U_{ab} \) mit dem Digitalmultimeter erfasst.

Bei der zweiten Messung wird die Versorgungspannung so gewählt,
dass die Brückenspannung \( U_{ab} \) proportional zu der angehängten Last steigt.
Diese Spannung wurde durch ausprobieren iterativ bestimmt.

\section[Messdaten]{Messdaten}

\begin{align*}
    R_a &= 1.1 \si{\mega\ohm}
\end{align*}
\begin{table}[!h]
    \centering
    \begin{tabular}{rrr}
        \toprule
            ~                 & \multicolumn{1}{c}{METRAHit TECH}    \\
            ~                 & \multicolumn{1}{c}{($U_{ab}$) Voltage}      \\
        \midrule
              0\si{\gram}      & -0.0005\si{\volt}  \\
            100\si{\gram}      & -0.0015\si{\volt}  \\
            200\si{\gram}      & -0.0024\si{\volt}  \\
            300\si{\gram}      & -0.0033\si{\volt}  \\
            400\si{\gram}      & -0.0043\si{\volt} \\
            500\si{\gram}      & -0.0052\si{\volt} \\
            ~ & ~ \\
        \bottomrule
        \end{tabular}
    \caption{Messdaten der Vollbrücke bei Belastung des Biegebalkens} \label{tab:full-bridge-voltage-measurement}
\end{table}
\begin{align*}
    R_a &= 1.1 \si{\mega\ohm}
\end{align*}
\begin{table}[!h]
    \centering
    \begin{tabular}{rrr}
        \toprule
            ~ & \multicolumn{1}{c}{METRAHit TECH} & \multicolumn{1}{c}{Source} \\
            ~ & \multicolumn{1}{c}{($U_{ab}$) Voltage} & \multicolumn{1}{c}{($U_{0}$) Voltage} \\
        \midrule
            100 \si{\gram} & 0.0014 \si{\volt} & 5.820 \si{\volt} \\
            200 \si{\gram} & 0.0023 \si{\volt} & 5.820 \si{\volt} \\
            300 \si{\gram} & 0.0032 \si{\volt} & 5.820 \si{\volt} \\
            400 \si{\gram} & 0.0041 \si{\volt} & 5.820 \si{\volt} \\
            500 \si{\gram} & 0.0050 \si{\volt} & 5.820 \si{\volt} \\
            ~ & ~ \\
        \bottomrule
        \end{tabular}
    \caption{Messdaten der Vollbrücke bei Anpassung von \(U_0\)} \label{tab:full-bridge-voltage-adjustment}
\end{table}

\newpage

\section[Auswertung]{Auswertung}
Ein Brückengleichgewicht kann bereits mit einem sehr geringen Strom
hergestellt werden, weshalb ein sehr großer Widerstand \( R_a = 1.1 \si{\mega\ohm} \) benötigt wird,
um einen Nullabgleich zu erzielen.
Zu sehen ist, dass die Spannung \( U_{ab} \) bei zunehmender Belastung steigt.
Dies ist auf die oben beschriebene Widerstandsveränderung
der Dehnmessstreifen durch Verformung des Messstreifens
im Dehnungsmessstreifen zurückzuführen.
Die Empfindlichkeit \( E_0 \) entspricht
\begin{align*}
    E_0 &= \frac{ U_0 }{ R_1 } = 0.0085 \si{\volt\per\ohm} \\
    \text{mit} \quad U_0 &= 6 \si{\volt} \\
    \text{und} \quad R_1 &= 700.7 \si{\ohm}
\end{align*}
wobei \( R_1 \) dem gemessenen Widerstand des Dehnmessstreifens aus Versuch 1 entspricht.
Zu sehen ist das bei einer Vollbrücke die Empfindlichkeit 4-mal so hoch ist wie bei einer Viertelbrücke, daraus kann man schließen das mit einer Vollbrücke eine präzisere Wage konstruiert werden kann.
Die Empfindlichkeit ist nicht von der angehangenen Belastung abhängig.