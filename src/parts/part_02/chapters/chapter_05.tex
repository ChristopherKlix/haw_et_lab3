\chapter[Wägeeinrichtung (Vollbrücke)]{Aufbau einer Wägeeinrichtung mit dem Biegestab (Vollbrücke)}

\lipsum[1]

\begin{figure}[!h]\centering
    \vspace*{0.7cm}
    \begin{circuitikz}[american, scale = 0.7]
    \draw
    (0,0) to[european voltage source, l=$U_0$] (0,6)

    (0,6) to[short, -*] (6,6)
          to[variable resistor, l=$R_1$] (6,3)

    (6,6) to[short, -*] (12,6)
          to[variable resistor, l=$R_3$] (12,3)

    (6,3) to[short, *-o] (7,3)
    (7,3) node[label=$a$] {}
    (12,3) to[short, *-o] (11,3)
    (11,3) node[label=$b$] {}

    (7,3) to[voltmeter, l=$U_{ab}$] (11,3)

    (6,3) to[variable resistor, l=$R_2$] (6,0)
    (12,3) to[variable resistor, l=$R_4$] (12,0)

    (0,0) to[short, -*] (6,0)
          to[short, -*] (12,0)

    % (2.5,6.2) node[label={[font=\footnotesize]east:$I_M$}] {}
    % (4.3,4) node[label={[font=\footnotesize]east:$U_M$}] {}

    % (7.5,4) node[label={[font=\footnotesize]east:Long distance}] {}
    ;
    \end{circuitikz}
    \caption{Wägeeinrichtung (Viertelbrücke)} \label{cir:quarter-bridge}
\end{figure}

\lipsum[1]

\section[Messung]{Messung}
