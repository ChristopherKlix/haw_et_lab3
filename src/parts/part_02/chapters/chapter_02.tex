\chapter[Wheatstonebrücke]{Bestimmung des Widerstandes eines Dehnungsmessstreifens nach dem Abgleichverfahren mit einer Wheatstonebrücke}

Die Brückenschaltung ist gemäß der Schaltung \ref{cir:ablgeich-wheatstone-bridge} aufzubauen.
Für \( R_1 \) ist der Dehnungsmessstreifen-Widerstand des unbelasteten Biegebalkens zu verwenden.
Der Widerstand \( R_2 \) ist mit einer Präzisionswiderstandsdekade \textit{Typ 4107} aufzubauen.
Für die Referenz-Widerstände \( R_3 \) und \( R_4 \) sind Präzisionswiderstände mit \( 1k\si{\ohm} \) aus dem \textit{hps Board} zu verwenden.

\begin{figure}[!h]\centering
    \vspace*{0.7cm}
    \begin{circuitikz}[american, scale = 0.7]
    \draw
    (0,0) to[european voltage source, l=$U_0$] (0,6)

    (0,6) to[short, -*] (6,6)
          to[resistor, l=$R_1$] (6,3)

    (6,6) to[short, -] (12,6)
          to[resistor, l=$R_3$] (12,3)

    (6,3) to[short, *-o] (7,3)
    (7,3) node[label=$a$] {}
    (12,3) to[short, *-o] (11,3)
    (11,3) node[label=$b$] {}

    (7,3) to[voltmeter, l=$U_{ab}$] (11,3)

    (6,3) to[potentiometer, l=$R_2$] (6,0)
    (12,3) to[resistor, l=$R_4$] (12,0)

    (0,0) to[short, -*] (6,0)
          to[short, -] (12,0)

    % (2.5,6.2) node[label={[font=\footnotesize]east:$I_M$}] {}
    % (4.3,4) node[label={[font=\footnotesize]east:$U_M$}] {}

    % (7.5,4) node[label={[font=\footnotesize]east:Long distance}] {}
    ;
    \end{circuitikz}
    \caption{Abgeglichene Wheatstonebrücke zur Bestimmung von $R_1$} \label{cir:ablgeich-wheatstone-bridge}
\end{figure}

Die Versorgunsspannung \( U_0 \) ist auf \( 6\si{\volt} \) einzustellen.
Die Spannung \( U_{ab} \) wird mit einem Multimeter \textit{METRAHit TECH} gemessen.
Die Brückenspannung \( U_{ab} \) ist durch Veränderung des Widerstandes \( R_2 \) über die Präzisions-Widerstandsdekade abzugleichen.
Der Widerstand \( R_1 \) ist unter den Abgleichbedingungen und der Kenntnis über \( R_2 \) und \( R_3 \) und \( R_4 \) zu bestimmen.

\section[Widerstandsmessung]{Widerstandsmessung mittels Wheatstonebrücke}
\subsection{Durchführung}

Die Brückenschaltung wurde gemäß der Schaltung \ref{cir:ablgeich-wheatstone-bridge} aufgebaut.
Die Widerstandsdekade \( R_2 \) wurde stufenweise verstellt,
bis das Multimeter einen Abgleich von \( 0\si{\volt} \) anzeigt hat.

\section{Messdaten}

\begin{align}
    R_2 = 699.6 \si{\ohm}
\end{align}

\section{Auswertung}

Aufgrund des Prinzips der Wheatstonebrücke und der Messung aus Versuch 1,
dass der unbelastete Widerstand des Dehnungsmessstreifens \( 700 \si{\milli\ohm } \) beträgt,
wurde vorher schon vermutet,
dass sich ein Abgleich bei \( R_2 \approx 700 \si{\ohm} \) einstellt.
Ein Abgleich der Messbrücke, i.e., \( U_{ab} = 0 \si{\volt} \),
wurde bei \( R_2 = 699.6 \si{\ohm} \) erreicht.
Die Abweichung zu Versuch 1 ergibt sich durch die Widerstandsmessung
über die Spannung (Messbrücke) und der Toleranz der Widerstände \( R_2 \) und \( R_3 \) und \( R_4 \).