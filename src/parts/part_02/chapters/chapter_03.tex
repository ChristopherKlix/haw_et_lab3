\chapter[Übertragungsfunktion]{Ermittlung der Übertragungsfunktion der Wheatstonebrücke}

In diesem Versuch wird eine Wheatstonebrücke auf ihre Empfindlichkeit und einen Linearitätsfehler bei unterschiedlichen Brückenverhältnissen untersucht.
Dies geschieht rechnerisch als auch messtechnisch mittels unterschiedlichen Präzisionswiderständen und variabel einstellbaren Widerstandsdekaden.

\begin{figure}[!h]\centering
    \vspace*{0.7cm}
    \begin{circuitikz}[american, scale = 0.7]
    \draw
    (0,0) to[european voltage source, l=$U_0$] (0,6)

    (0,6) to[short, -*] (6,6)
          to[variable resistor, l=$R_1$] (6,3)

    (6,6) to[short, -*] (12,6)
          to[resistor, l=$R_3$] (12,3)

    (6,3) to[short, *-o] (7,3)
    (7,3) node[label=$a$] {}
    (12,3) to[short, *-o] (11,3)
    (11,3) node[label=$b$] {}

    (7,3) to[voltmeter, l=$U_{ab}$] (11,3)

    (6,3) to[variable resistor, l=$R_2$] (6,0)
    (12,3) to[resistor, l=$R_4$] (12,0)

    (0,0) to[short, -*] (6,0)
          to[short, -*] (12,0)

    % (2.5,6.2) node[label={[font=\footnotesize]east:$I_M$}] {}
    % (4.3,4) node[label={[font=\footnotesize]east:$U_M$}] {}

    % (7.5,4) node[label={[font=\footnotesize]east:Long distance}] {}
    ;
    \end{circuitikz}
    \caption{Wheatstonebrücke} \label{cir:wheatstone-bridge}
    \end{figure}

Die Brückenschaltung wird nach Schaltskizze \ref{cir:wheatstone-bridge} aufgebaut.
Die Widerstände \( R_1 \) und \( R_2 \) werden mit 2 Präzisionswiderstandsdekaden \textit{Typ 4107} aufgebaut.
Die Widerstände \( R_3 \) und \( R_4 \) werden mit Fixwiderständen mit \( 0.02\% \) Toleranz aufgebaut.
Die Versorgunsspannung beträgt \( U_0 = 6 \si{\volt} \).
Die Spannung \( U_{ab} \) wird mit einem Multimeter \textit{METRAHit Tech} gemessen.

\section[Messung 1]{Messung 1}
\subsection{Durchführung}

Im ersten Messdurchlauf werden für die Widerstandsverhältnis
\begin{align}
    a_1 = \frac{R_4}{R_3} = \frac{1 \si{\kilo\ohm}}{1 \si{\kilo\ohm}}
\end{align}
gewählt.
Der Widerstand \( R_1 \) wird an der Widerstandsdekade auf \( 700 \si{\ohm} \).
Der Widerstand \( R_2 \) wird so eingestellt, dass die Brücke abgeglichen ist,
also die Spannung \( U_{ab} \) möglichst \( 0 \si{\volt} \) anzeigt.
Nun wird \( R_1 \) auf \( 400 \si{\ohm} \) gestellt und in \( 100 \si{\ohm} \) Schritten bis \( 1.3 \si{\kilo\ohm} \) gesteigert.
Es wird die jeweilige Spannung \( U_{ab} \) erfasst.

\section[Messung 2]{Messung 2}
\subsection{Durchführung}

Im zweiten Messdurchlauf werden für die Widerstandsverhältnis
\begin{align}
    a_2 = \frac{R_4}{R_3} = \frac{100 \si{\ohm}}{1 \si{\kilo\ohm}}
\end{align}
gewählt.
Der Widerstand \( R_1 \) wird an der Widerstandsdekade auf \( 700 \si{\ohm} \).
Der Widerstand \( R_2 \) wird so eingestellt, dass die Brücke abgeglichen ist,
also die Spannung \( U_{ab} \) möglichst \( 0 \si{\volt} \) anzeigt.
Nun wird \( R_1 \) auf \( 400 \si{\ohm} \) gestellt und in \( 100 \si{\ohm} \) Schritten bis \( 1.3 \si{\kilo\ohm} \) gesteigert.
Es wird die jeweilige Spannung \( U_{ab} \) erfasst.

\section[Messdaten]{Messdaten}

\begin{align*}
    a_1 = 1 \quad \text{und} \quad a_2 = 0.1
\end{align*}

\begin{table}[!h]
    \centering
    \begin{tabular}{rrr}
    \toprule
        ~                 & \multicolumn{2}{c}{METRAHit TECH}    \\
        ~                 & \multicolumn{2}{c}{(U) Voltage}      \\
    \midrule

    \rowcolor{Gray}
        ~                 & [$a_1$] Messung 1  & [$a_2$] Messung 2         \\
        400\si{\ohm}      & 0.820\si{\volt}  & 0.349\si{\volt}   \\
        500\si{\ohm}      & 0.500\si{\volt}  & 0.192\si{\volt}   \\
        600\si{\ohm}      & 0.231\si{\volt}  & 0.082\si{\volt}   \\
        700\si{\ohm}      & 0.000\si{\volt}  & 0.001\si{\volt}   \\
        800\si{\ohm}      & -0.200\si{\volt} & -0.062\si{\volt}  \\
        900\si{\ohm}      & -0.375\si{\volt} & -0.112\si{\volt}  \\
        1\si{\kilo\ohm}   & -0.530\si{\volt} & -0.153\si{\volt}  \\
        1.1\si{\kilo\ohm} & -0.667\si{\volt} & -0.186\si{\volt}  \\
        1.2\si{\kilo\ohm} & -0.790\si{\volt} & -0.215\si{\volt}  \\
        1.3\si{\kilo\ohm} & -0.901\si{\volt} & -0.244\si{\volt}  \\
        ~ & ~ \\
    \bottomrule
    \end{tabular}
    \caption{Spannungsmessung der Wheatstonebrücke.}
    \label{wheatstonebridge-voltage-measurement}
\end{table}

\begin{figure}
    \centering
    \begin{tikzpicture}
        \begin{axis}
            [
                xlabel={[R] Resistance},
                ylabel={[U] Voltage},
                grid,
                xmin=400,
                xmax=1300,
                ymin=-1,
                ymax=1,
                samples = 50,
                domain=400:1300,
                legend pos=north east,
                tick align=inside
            ]
            \addplot[cyan, no marks, smooth, ultra thick] table[col sep=comma, x=resistance, y=voltage]{./data/measurement_01.csv};
            \addplot[red, no marks, smooth, ultra thick] table[col sep=comma, x=resistance, y=voltage]{./data/measurement_02.csv};

            \addlegendentry{[$a_1$] Messung 1}
            \addlegendentry{[$a_2$] Messung 2}

        \end{axis}
    \end{tikzpicture}
    \caption{Plot der Spannungsmessung der Wheatstonebrücke.}
    \label{fig:wheatstonebridge-voltage-measurement}
\end{figure}

\newpage

\section[Berechnungen]{Berechnungen}

\begin{align}
    U_{ab} = U_0 \cdot \frac{a}{(1 + a) ^ 2} \cdot \frac{\Delta R_1}{R_1}
\end{align}

\begin{align}
    a_1 &= \frac{1 \si{\kilo\ohm}}{1 \si{\kilo\ohm}} \quad \Rightarrow R_1 = 700 \si{\ohm}\\
    a_2 &= \frac{100 \si{\ohm}}{1 \si{\kilo\ohm}} \quad \Rightarrow R_1 = 70 \si{\ohm}
\end{align}

\begin{table}[h]
    \begin{subtable}[h]{0.45\textwidth}
        \centering
        \begin{tabular}{rrr}
            \toprule
                ~        & \multicolumn{1}{c}{300 \si{\ohm}} & \multicolumn{1}{c}{600 \si{\ohm}}   \\
            \midrule
                $U_{ab}$ & 0.643\si{\volt}  & 1.286\si{\volt}   \\
                ~ & ~ \\
            \bottomrule
        \end{tabular}
        \caption{Spannungsberechnung für $a_1$.}
        \label{tab:wheatstonebridge-voltage-calculation-a1}
    \end{subtable}
    \begin{subtable}[h]{0.45\textwidth}
        \centering
        \begin{tabular}{rrr}
            \toprule
                ~        & \multicolumn{1}{c}{330 \si{\ohm}} & \multicolumn{1}{c}{1230 \si{\ohm}}   \\
            \midrule
                $U_{ab}$ & 2.34\si{\volt}  & 8.71\si{\volt}   \\
                ~ & ~ \\
            \bottomrule
        \end{tabular}
        \caption{Spannungsberechnung für $a_2$.}
        \label{tab:wheatstonebridge-voltage-calculation-a2}
    \end{subtable}
    \caption{Spannungsberechnung der Wheatstonebrücke.}
    \label{tab:wheatstonebridge-voltage-calculation}
\end{table}

\newpage

\section[Auswertung]{Auswertung}

Betrachtet man die Graphen, ist zu erkennen, dass der Graph für das Brückenverhältnis \( a = 1 \)
deutlich steiler verläuft, als der Graph zum Brückenverhältnis \( a = 0.1 \).
Daraus ist zu schließen, dass ein Brückenverhältnis von \( a = 1 \),
empfindlicher gegenüber einer Veränderung von \( R_1 \) ist als ein Brückenverhältnis von \( a = 0.1 \).
Also ist hier eine größere Spannungsänderung an \( U_{ab} \) zu messen, pro Ohm Widerstandsänderung von \( R_1 \).
Die Berechnung mittels der Näherungs-Formel mag für kleine Verstimmungen
\[ \frac{\Delta R}{R_1} \]
ausreichend genau sein,
doch weißt zu große Fehler bei derart großen Verstimmungen auf.
Dies liegt daran, dass die Formel eine Linearität annimmt und lediglich die werte entlang einer Tangente,
angelegt an den Abgleichpunkt berechnet.
Wählt man also große Verstimmungen für die Berechnung, wird man auch einen großen Linearitätsfehler erhalten.