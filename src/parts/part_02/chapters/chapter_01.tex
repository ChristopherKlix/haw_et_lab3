\chapter[Dehnungsmessstreifen]{Bestimmung des Widerstandes eines Dehnungsmessstreifens}

In diesem Versuch wird ein auf einem Biegebalken angebrachten Dehnungsmessstreifen untersucht.
Der Widerstand des Dehnungsmessstreifen wird einmal im unbelasteten und einmal im belsteten Zustand gemessen.

\section[Ohmmeter]{Widerstandsmessung mittels Ohmmeter}
\subsection{Durchführung}

Die Belastung erfolgt durch das Gewicht eines \( 200\si{\gram} \) schweren Körpers,
welcher an die Spitze des Biegebalkens gehängt wird.
Die Widerstandsmessung erfolgt mittels eines digitalen Multimeters \textit{METRAHit 18S}.

\subsection{Messdaten}

\begin{table}[!h]
    \centering
    \begin{tabular}{rr}
    \toprule
        ~                 & METRAHit 18S  \\
        Weight            &               \\
    \midrule
        0\si{\gram}       & 700.7\si{\ohm} \\
        ~ & ~ \\
        200\si{\gram}     & 700.8\si{\ohm} \\
    \bottomrule
    \end{tabular}
    \caption{\label{multimeter-resistor-measurement}Widerstandsmessung mittels Multimeter.}
\end{table}

\subsection{Auswertung}

Es wurde eine Widerstandsänderung von \( \Delta R = +100 \si{\milli\ohm} \) gemessen.
Diese Erhöhung des Widerstandes lässt durch die Änderung der materialen Abmessungen innerhalb des Dehnstreifens beschreiben.
Denn der auf der Oberseite des Biegebalkens befestigte Dehnmessstreifen wird durch die Biegung des Balkens gestreckt.
Dies führt zu einer Verlängerung bzw. Dehnung des Drahtes im Dehnessstreifen.
Diese Verlängerung führt zusätzlich zu einer Querkontraktion, also einer Verkleinerung des Querschnittes.
Nach der Formel \( R= \rho \cdot \frac{l}{A}\) ergibt sich dann ein erhöhter Widerstandswert,
da die Querschnittsfläche \( A \) im Nenner verkleinert wird und die Leiter Länge \( l \) im Zähler vergrößert.