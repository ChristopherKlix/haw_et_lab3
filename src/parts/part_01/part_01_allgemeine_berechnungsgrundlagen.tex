\part{Allgemeine Berechnungsgrundlagen}

\chapter*{Allgemeine Berechnungsgrundlagen}

\section*{Konzepte}
\addcontentsline{toc}{section}{Konzepte}

\begin{itemize}
  \item Grundlagen der Netzwerkanalyse
  \item Ermittlung einer linearen Ersatzspannungsquelle
  \begin{itemize}
    \item Ermittlung des Innenwiderstandes \(R_i\)
    \item Spannung der idealen Ersatzspannungsquelle \(U_{ab}\) = Spannung zwischen den Messpunkten \(a\) \& \(b\).
    \item Leistungsanpassung
  \end{itemize}
  \item Kirchoff’schen Gesetze
  \begin{itemize}
    \item Knotenregel:
    \begin{description}
        \item Die Summe aller ein und ausfliesenden Ströme in einem Knoten sind Null.
    \end{description}
    \item Maschenregel
    \begin{description}
        \item Die Summe aller Spannungen entlang eines Maschenumlaufes ist
gleich Null.
    \end{description}
  \end{itemize}
  \item Superpositionsprinzip in Schaltkreisen
\end{itemize}

\section*{Formeln}
\addcontentsline{toc}{section}{Formeln}

\subsection*{Ohm'sche Gesetz}

\[
U = R \cdot I
\]

\subsection*{Widerstände in Reihe}

\[
\sum_{i=1}^{n} R_i = R_{ges}
\]

\subsection*{Widerstände in Parallel}

\[
\sum_{i=1}^{n} \frac{1}{R_i} = R_{ges}^{-1}
\]

\[
\left[ \sum_{i=1}^{n} \frac{1}{R_i} \right]^{-1} = R_{ges}
\]

\subsection*{Spannungsteiler}

\[
U_i = U_0 \cdot \frac{R_i}{R_{ges}}
\]

\subsection*{Leistungsanpassung für lineare Ersatzspannungsquelle}
Leistung ist maximal, wenn \(R_i\) gleich \(R_L\) ist.

\[
P_{max} = \frac{U_0^{2}}{4R_i}
\]

\subsection*{Widerstandsmessung - Relativer Fehler bei stromrichtiger Messung}

\[
e_{rel} \approx \frac{R_{i_A}}{R_x}
\]
wobei \(R_x\) der zu messende Widerstand ist.

\subsection*{Widerstandsmessung - Relativer Fehler bei spannungsrichtiger Messung}

\[
e_{rel} \approx -\frac{R_x}{R_{i_V}}
\]
wobei \(R_x\) der zu messende Widerstand ist.




\renewcommand{\chaptername}{Versuch}
\newpage
\pagenumbering{arabic}
