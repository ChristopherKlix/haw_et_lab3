\part{Allgemeine Berechnungsgrundlagen}

\chapter*{Allgemeine Berechnungsgrundlagen}

\section*{Konzepte}
\addcontentsline{toc}{section}{Konzepte}

\begin{itemize}
  \item Grundlagen der Netzwerkanalyse
  \item Ermittlung einer linearen Ersatzspannungsquelle
  \begin{itemize}
    \item Ermittlung des Innenwiderstandes \(R_i\)
    \item Spannung der idealen Ersatzspannungsquelle \(U_{ab}\) = Spannung zwischen den Messpunkten \(a\) \& \(b\).
    \item Leistungsanpassung
  \end{itemize}
  \item Kirchoff’schen Gesetze
  \begin{itemize}
    \item Knotenregel:
    \begin{description}
        \item Die Summe aller ein und ausfliesenden Ströme in einem Knoten sind Null.
    \end{description}
    \item Maschenregel
    \begin{description}
        \item Die Summe aller Spannungen entlang eines Maschenumlaufes ist
gleich Null.
    \end{description}
  \end{itemize}
  \item Superpositionsprinzip in Schaltkreisen
\end{itemize}

\section*{Formeln}
\addcontentsline{toc}{section}{Formeln}

\subsection*{Ohm'sche Gesetz}

\begin{align}
    U = R \cdot I
\end{align}

\subsection*{Widerstände in Reihe}

\begin{align}
    \sum_{i=1}^{n} R_i = R_{ges}
\end{align}

\subsection*{Widerstände in Parallel}

\begin{align}
    \begin{split}
        \sum_{i=1}^{n} \frac{1}{R_i} = R_{ges}^{-1} \\
        \left[ \sum_{i=1}^{n} \frac{1}{R_i} \right]^{-1} = R_{ges}
    \end{split}
\end{align}

\subsection*{Spannungsteiler}

\begin{align}
    U_i = U_0 \cdot \frac{R_i}{R_{ges}}
\end{align}

\subsection*{Leistungsanpassung für lineare Ersatzspannungsquelle}
Leistung ist maximal, wenn \(R_i\) gleich \(R_L\) ist.

\begin{align}
    P_{max} = \frac{U_0^{2}}{4R_i}
\end{align}

\subsection*{Widerstandsmessung - Relativer Fehler bei stromrichtiger Messung}

\begin{align}
    e_{rel} \approx \frac{R_{i_A}}{R_x}
\end{align}
wobei \(R_x\) der zu messende Widerstand ist.

\subsection*{Widerstandsmessung - Relativer Fehler bei spannungsrichtiger Messung}

\begin{align}
    e_{rel} \approx -\frac{R_x}{R_{i_V}}
\end{align}
wobei \(R_x\) der zu messende Widerstand ist.

\newpage

\subsection*{Brückenempfindlichkeit}

Änderung von \( U_{ab} \) in Abhängigkeit von \( R_1 \) im Ablgeichpunkt.

\begin{align}
    E_0 &= \frac{ \text{d} U_{ab} }{ \text{d} R_1 }
    \approx \frac{ \Delta U_{ab} }{ \Delta R_1 }
    = \frac{ U_{ab} }{ \Delta R_1 }
    \quad \text{für } \frac{R_1}{R_2} = \frac{R_3}{R_4}
\end{align}

Für \( \Delta R_1 \ll R_1 \) ergibt sich mit:

\begin{align}
    \begin{split}
        \frac{ U_{ab} }{ U_0 } &\approx \frac{ a }{ \left( 1 + a \right)^2 } \cdot \frac{ \Delta R_1 }{ R_1 } \\
        \iff \quad U_{ab} &\approx \frac{ a }{ \left( 1 + a \right)^2 } \cdot \frac{ \Delta R_1 }{ R_1 } \cdot U_0
    \end{split}
\end{align}

Daher:

\begin{align}
    \begin{split}
        E_0 &= \frac{ a }{ \left( 1 + a \right)^2 } \cdot \frac{ U_0 }{ R_1 }
    \end{split}
\end{align}

\subsection*{Brückenverhältnis}

\begin{align}
    a &= \frac{ R_2 }{ R_1 } = \frac{ R_4 }{ R_3 }
\end{align}

\subsection*{Relative Verstimmung der Brücke}

\begin{align}
    v &= \frac{ \Delta R_1 }{ R_1 }
\end{align}
