\part{Zielsetzung}

In dieser Versuchsreihe wird das Verhalten von Brücken untersucht.
Die Brücke ist ein elektrisches Element, das eine Spannung proportional zu einem Strom oder einem Widerstand misst.
Die Brücke ist ein wichtiger Bestandteil von elektronischen Messgeräten.
Mittels verschiedener Methoden und Aufbauten werden die Eigenschaften des zu messenden Widerstandes untersucht.
Die Messung des Widerstandes erfolgt zunächst mittels eines Multimeters.
Darauf aufbauend wird die Messung mittels einer Brücke mit einem Präzisionswiderstand durchgeführt.
Ziel dieser Veruche ist es, die Messgenauigkeit und Brückenempfindlichkeit der Brücke zu untersuchen und zu verstehen.
Als zu messender Widerstand wird ein Dehnungsmessstreifen verwendet.